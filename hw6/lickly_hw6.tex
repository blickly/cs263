\documentclass{article}
\usepackage{amsmath}
\usepackage{amsthm}
%\usepackage{fullpage}
\usepackage{stmaryrd}
\usepackage{graphicx}
\usepackage{algorithmic}
\usepackage{wrapfig}

\title{CS263: HW \#6}
\author{Ben Lickly}
\date{March 30, 2010}

\newcommand{\problem}[1]
{\subsubsection*{} %\fbox{\parbox{\textwidth}{
\vspace{-16pt} \section{} \vspace{-22pt} \qquad
#1%}}
\bigskip \bigskip
}

\newcommand{\while}[2]{\operatorname{while}\, #1\ \operatorname{do}\ #2}
\newcommand{\ifthen}[3]{\operatorname{if}\, #1\ \operatorname{then}\ #2\ \operatorname{else}\ #3}
\newcommand{\denote}[1]{\llbracket #1 \rrbracket}
\newcommand{\proves}{\vdash}
\newcommand{\axiomatic}[3]{\{#1\}\ #2\ \{#3\}}
\newcommand{\meet}{\bigwedge}
\newcommand{\powerset}{\mathcal{P}}
\newcommand{\glb}{\operatorname{glb}}

\begin{document}
\maketitle

\problem{Show the encoding of the predecessor function in lambda-calculus, using
the encoding of natural numbers that we have given in class. The predecessor of 0
should be 0. Please do not write your solution as a big mess of lambdas. Use
intermediate functions (like we did for $add$ and $succ$) that make it easy to
understand how your encoding works.
}

For this problem, I will create a new numeric type called $newnat$ that extends
the old natural numbers with a notion for $-1$.  The idea is then to define
predecessor as applying the $newnat$'s successor function to $-1$ $n$ times.

In the following definitions, the argument $n$ will refer to the old style of
natural numbers and the argument $m$ will refer to the new type.
\begin{align*}
newneg1 = mkpair true 0 \\
newnat n = mkpair false n \\
oldnat m = snd m \\
newsucc m = (fst m) (newnat 0) (newnat (succ (oldnat m))) \\
pred n = oldnat (n newsucc newneg1)
\end{align*}

\problem{Show how you can encode lists in lambda-calculus. Show the encoding of
$nil$ (empty list), $cons$ (prepend an element to a list), $length$ (return the
natural number representing the length of a list), and $append$. Please keep it
simple.
}

We can play the same trick that numbers do where a list is an object that we
give a function and an object and it repeatedly applies the function. This
time, however, the function will also need to look at the contents of the list,
so it will be a two argument function rather than a one argument one.

\[
nil = 0
\]

\[
cons x l = \lambda f \lambda s.  f x (l f s)
\]


\[
length l = l (\lambda x \lambda y. succ y) 0
\]

\[
append l1 l2 = \lambda f \lambda s. l1 f (l2 f s)
\]

\problem{Show that the execution always terminates in the call-by-value
simply-typed $\lambda$-calculus. You need to consider only variables, integer
constants, addition, abstraction and application. If you use induction, state
precisely on what you induct.
}

\end{document}