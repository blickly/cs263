\documentclass{article}
\usepackage{amsmath}
\usepackage{amsthm}
%\usepackage{fullpage}
\usepackage{stmaryrd}
\usepackage{graphicx}
\usepackage{algorithmic}
\usepackage{wrapfig}

\title{CS263: HW \#7}
\author{Ben Lickly}
\date{April 13, 2010}

\newcommand{\problem}[1]
{\subsubsection*{} %\fbox{\parbox{\textwidth}{
\vspace{-16pt} \section{} \vspace{-22pt} \qquad
#1%}}
\bigskip \bigskip
}

\newcommand{\while}[2]{\operatorname{while}\, #1\ \operatorname{do}\ #2}
\newcommand{\ifthen}[3]{\operatorname{if}\, #1\ \operatorname{then}\ #2\ \operatorname{else}\ #3}
\newcommand{\denote}[1]{\llbracket #1 \rrbracket}
\newcommand{\proves}{\vdash}
\newcommand{\axiomatic}[3]{\{#1\}\ #2\ \{#3\}}
\newcommand{\meet}{\bigwedge}
\newcommand{\powerset}{\mathcal{P}}
\newcommand{\glb}{\operatorname{glb}}

\begin{document}
\maketitle

\problem{What is the type of the combinators $K$ and $S$ in the polymorphic
lambda calculus $F_2$ ? Argue that the combinator $D$ does not have a type in
$F_2$.
}

The $K$ combinator is the unique function (in the pure $F_2$
lambda calulus) with type
$\Lambda t_1. \Lambda t_2. t_1 \to t_2 \to t_1$.

The $S$ combinator has type
$\Lambda t_1. \Lambda t_2. \Lambda t_3.
(t_1 \to t_2 \to t_3) \to (t_1 \to t_2) \to t_1 \to t_3$

The $D$ combinator can not have a type because it would need to satisfy the
equation $t(t) = t$.  We may think naively that the type $\Lambda t. t \to t$
would work here, but this allows any $t$, whereas we need to restrict ourselves
only to function types (whose argument and return are both function types whose
argument and return are \dots).  We may then try
$\Lambda t. (t \to t) \to (t \to t)$, but we will run into the same problem one
level deeper.  It turns out that no finite $F_2$ type expression can express
the need that there are functions ``all the way down''.

\problem{What would be a good subtyping rule for continuations?  That is, under
what conditions it is sound to say that $T_1 cont \le T_2 cont$.
}

If a program only uses $callcc$, then it's continuations are covariant and we
can say that $T_1 \le T_2 \implies T_1 cont \le T_2 cont$.

If a program only uses $throw$, then it's continuations are contravariant, and
we can say that $T_2 \le T_1 \implies T_1 cont \le T_2 cont$.

In the general case, however, we can not say that continuations are either
covariant or contravariant.

\end{document}
