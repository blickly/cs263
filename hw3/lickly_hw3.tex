\documentclass{article}
\usepackage{amsmath}
%\usepackage{fullpage}
\usepackage{stmaryrd}

\title{CS263: HW \#3}
\author{Ben Lickly}
\date{February 16, 2010}

\newcommand{\problem}[1]
{\subsubsection*{} %\fbox{\parbox{\textwidth}{
\vspace{-16pt} \section{} \vspace{-22pt} \qquad
#1%}}
\bigskip \bigskip
}

\newcommand{\while}[2]{\operatorname{while}\, #1\ \operatorname{do}\ #2}
\newcommand{\ifthen}[3]{\operatorname{if}\, #1
\ \operatorname{then}\#2\ \operatorname{else}\ #3}
\newcommand{\denote}[1]{\llbracket #1 \rrbracket}

\begin{document}
\maketitle
\problem{
Prove the following statement. For any boolean command $b$ and any initial state
$\sigma$ such that $\sigma(x)$ is even, if $\denote{\while{b}{x := x + 2}}
\sigma = \sigma'$ and if $\sigma = \bot$ then $\sigma'(x)$ is even. Unlike in previous assignment, this time you should use denotational semantics
for the proof.
}

Let $w = \while{b}{x := x +1}$ and $W = \denote{w}.$

Case: the rule at the root is while-false
\[
W(\sigma) = \begin{cases}
            \bot    &\text{if } \forall k, W_k(\sigma) = \bot \\
            \sigma' &\text{if } W_k(\sigma) = \sigma' \ne \bot
            \end{cases}
\]
\[
W_k(\sigma) = \ifthen{\denote{b}\sigma}{W_{k-1}(\denote{c}\sigma)}{\sigma}
\]
\[
W(\sigma) = \ifthen{\denote{b}\sigma}{W(\denote{c}\sigma)}{\sigma}
\]


\problem{
Recall that we obtained the least fixed point of a monotone
and continuous function $F : D \to D$ by starting with $\bot = F^0(\bot)$ and
repeatedly applying $F$ to the result. Let's explore what happens if we do not
start with $\bot$ but with some other element $x \in D$.
\begin{enumerate}
  \item What is a necessary and sufficient condition for the sequence $F^i(x)$
  to be a chain, that is $x \sqsubseteq F(x) \sqsubseteq F^2 (x) \ldots$ ?

  \item Prove that if the condition above holds then $\bigsqcup_i F^i (x)$ is a
  fixed-point and furthermore it is the least of all fixed-points that are
  $\sqsupseteq$ than $x$.

  \item Find a necessary and sufficient condition for the $\bigsqcup_i F^i (x)$ to be
the overall least-fixed point.
\end{enumerate}
}



\end{document}