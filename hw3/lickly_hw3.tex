\documentclass{article}
\usepackage{amsmath}
%\usepackage{fullpage}
\usepackage{stmaryrd}

\title{CS263: HW \#3}
\author{Ben Lickly}
\date{February 16, 2010}

\newcommand{\problem}[1]
{\subsubsection*{} %\fbox{\parbox{\textwidth}{
\vspace{-16pt} \section{} \vspace{-22pt} \qquad
#1%}}
\bigskip \bigskip
}

\newcommand{\while}[2]{\operatorname{while}\, #1\ \operatorname{do}\ #2}
\newcommand{\ifthen}[3]{\operatorname{if}\, #1
\ \operatorname{then}\ #2\ \operatorname{else}\ #3}
\newcommand{\denote}[1]{\llbracket #1 \rrbracket}

\begin{document}
\maketitle
\problem{
Prove the following statement. For any boolean command $b$ and any initial state
$\sigma$ such that $\sigma(x)$ is even, if $\denote{\while{b}{x := x + 2}}
\sigma = \sigma'$ and if $\sigma' \ne \bot$ then $\sigma'(x)$ is even. Unlike in
previous assignment, this time you should use denotational semantics for the proof.
}

Let $w = \while{b}{x := x +2}$ and $W = \denote{w}$.
%
Thus, we are given the following:
\begin{itemize}
  \item $\exists n,\ \sigma(x) = 2n$
  \item $W(\sigma) = \sigma'$
  \item $\sigma' \ne \bot$
\end{itemize}
From these, we want to prove that
\[
\exists n',\ \sigma'(x) = 2n'
\]

\iffalse
Case: the rule at the root is while-false

\[
W_k(\sigma) = \ifthen{\denote{b}\sigma}{W_{k-1}(\denote{c}\sigma)}{\sigma}
\]
\[
W(\sigma) = \ifthen{\denote{b}\sigma}{W(\denote{c}\sigma)}{\sigma}
\]
\fi

The most intesesting case is that for which $B\denote{b}\sigma$ is true.
In this case, we will need to rely on our definition of $W$ and $W_k$:
\[
W(\sigma) = \begin{cases}
            \bot    &\text{if } \forall k, W_k(\sigma) = \bot \\
            \sigma' &\text{if } W_k(\sigma) = \sigma' \ne \bot
            \end{cases}
\]
\[
W_k(\sigma) =
\begin{cases}
\bot & \text{if } k = 0 \\
\ifthen{\denote{b}\sigma}{W_{k-1}(\denote{c}\sigma)}{\sigma} &\text{otherwise}
\end{cases}
\]
Here we note that in all the cases for which $W(\sigma) \ne \bot$, there exists
a $k$ for which $W = W_k$.  We will now do mathematical induction on $k$.

Base case: Since $W_0 = \bot$ it does not satisfy our proof assumptions.
The simplest interesting case is $W_1$.  The only case for which $W_1(\sigma) =
\sigma' \ne \bot$ is if $B\denote{b}\sigma$ is false.  Then we have that
$\sigma' = \sigma$, and we can finish our proof by simply setting $n'$ to $n$.

Inductive case:
Assume that we have proved all our theorem for $W_{k-1}$ with $k > 1$.  If
$B\denote{b}\sigma$ were false, we would be done, so the only interesting case
is when $B\denote{b}\sigma$ is true.
We want to prove that for $\sigma' = W_k(\sigma)$, if $\sigma' \ne
\bot$ then $\sigma'(x)$ is eve. In this case, we know that $W_k(\sigma) =
W_{k-1}(\denote{c}\sigma) = W_{k-1}(\sigma[x := x + 2])$.
Clearly, since $\sigma(x)$ is even, so is $\sigma[x := x + 2](x)$.
Thus, we can use our induction hypothesis to conclude that if $\sigma' \ne \bot$
then $\sigma'(x)$ is even, finishing our induction step.

\problem{
In this exercise you will prove using denotational semantics that the command
``$\while{b}{c}$'' is equivalent to ``$\ifthen{b}{c;\while{b}{c}}{skip}$''.
Essentially this means that you have to prove that the denotational semantic
function for while (slide 14 in the lecture) does satisfy the unwinding equation
for any state initial state.
}

Our denational definition of $\while{b}{c}$ is given as follows:
\[
W(\sigma) = \begin{cases}
            \bot    &\text{if } \forall k, W_k(\sigma) = \bot \\
            \sigma' &\text{if $k$ smallest such that } W_k(\sigma) = \sigma' \ne
            \bot
            \end{cases}
\]
\[
W_k(\sigma) =
\begin{cases}
\sigma' & \text{if $w$ in state $\sigma$ terminates in state $\sigma'$ in fewer
than $k$ iterations} \\
\bot &\text{otherwise}
\end{cases}
\]

We want to prove that $W(\sigma) =
\ifthen{B\denote{b}\sigma}{W(\denote{c}\sigma)}{\sigma}$ for all initial states
$\sigma$. We will prove this in two cases, depending on the truth value of
$B\denote{b}\sigma$.

Case 1: $B\denote{b}\sigma$ false \\
In this case we know that $W(\sigma) = W_1(\sigma) = \sigma \ne \bot$.
Simplifiying the right hand side also gives us $\sigma$.

Case 2:  $B\denote{b}\sigma$ true \\
In this case, the right hand side simplifies to $W(\denote{c}\sigma)$.
Here, there are two sub-cases: either the while loop never terminates (i.e.
$\forall k, W_k(\sigma) = \bot$ ), or it terminates after some finite number of
iterations (i.e. $\exists k, W_k(\sigma) = \sigma' \ne \bot$).

Case 2a: $\forall k, W_k(\sigma) = \bot$ \\
This means that $W(\sigma) = \bot$.
If $W(\denote{c}\sigma) = W_{k'}(\denote{c}\sigma) \ne \bot$, this would mean
that $W(\sigma)$ is also defined (by, for example $W_{k'+1}(\sigma)$).
Thus, it must also be the case that
$W(\denote{c}\sigma) = \bot$.

Case 2b: $\exists k, W_k(\sigma) = \sigma' \ne \bot$ \\
In this case, $W(\sigma) = W_k(\sigma)$.  After performing one iteration of
$c$, there will only be $k-1$ iterations remaining, so $W_k(\sigma) =
W_{k-1}(\denote{c}\sigma)$. This must be equal to $W(\denote{c}\sigma)$.

\end{document}