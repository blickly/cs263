\documentclass{article}
\usepackage{amsmath}
%\usepackage{fullpage}
\usepackage{stmaryrd}

\title{CS263: HW \#3}
\author{Ben Lickly}
\date{February 16, 2010}

\newcommand{\problem}[1]
{\subsubsection*{} %\fbox{\parbox{\textwidth}{
\vspace{-16pt} \section{} \vspace{-22pt} \qquad
#1%}}
\bigskip \bigskip
}

\newcommand{\while}[2]{\operatorname{while}\, #1\ \operatorname{do}\ #2}
\newcommand{\ifthen}[3]{\operatorname{if}\, #1
\ \operatorname{then}\ #2\ \operatorname{else}\ #3}
\newcommand{\denote}[1]{\llbracket #1 \rrbracket}

\begin{document}
\maketitle
\problem{
Prove the following statement. For any boolean command $b$ and any initial state
$\sigma$ such that $\sigma(x)$ is even, if $\denote{\while{b}{x := x + 2}}
\sigma = \sigma'$ and if $\sigma' \ne \bot$ then $\sigma'(x)$ is even. Unlike in
previous assignment, this time you should use denotational semantics for the proof.
}

Let $w = \while{b}{x := x +2}$ and $W = \denote{w}$.
%
Thus, we are given the following:
\begin{itemize}
  \item $\exists n,\ \sigma(x) = 2n$
  \item $W(\sigma) = \sigma'$
  \item $\sigma' \ne \bot$
\end{itemize}
From these, we want to prove that
\[
\exists n',\ \sigma'(x) = 2n'
\]

\iffalse
Case: the rule at the root is while-false

\[
W_k(\sigma) = \ifthen{\denote{b}\sigma}{W_{k-1}(\denote{c}\sigma)}{\sigma}
\]
\[
W(\sigma) = \ifthen{\denote{b}\sigma}{W(\denote{c}\sigma)}{\sigma}
\]
\fi

The most intesesting case is that for which $B\denote{b}\sigma$ is true.
In this case, we will need to rely on our definition of $W$ and $W_k$:
\[
W(\sigma) = \begin{cases}
            \bot    &\text{if } \forall k, W_k(\sigma) = \bot \\
            \sigma' &\text{if } W_k(\sigma) = \sigma' \ne \bot
            \end{cases}
\]
\[
W_k(\sigma) =
\begin{cases}
\bot & \text{if } k = 0 \\
\ifthen{\denote{b}\sigma}{W_{k-1}(\denote{c}\sigma)}{\sigma} &\text{otherwise}
\end{cases}
\]
Here we note that in all the cases for which $W(\sigma) \ne \bot$, there exists
a $k$ for which $W = W_k$.  We will now do mathematical induction on $k$.

Base case: Since $W_0 = \bot$ it does not satisfy our proof assumptions.
The simplest interesting case is $W_1$.  The only case for which $W_1(\sigma) =
\sigma' \ne \bot$ is if $B\denote{b}\sigma$ is false.  Then we have that
$\sigma' = \sigma$, and we can finish our proof by simply setting $n'$ to $n$.

Inductive case:
Assume that we have proved all our theorem for $W_{k-1}$ with $k > 1$.  If
$B\denote{b}\sigma$ were false, we would be done, so the only interesting case
is when $B\denote{b}\sigma$ is true.
We want to prove that for $\sigma' = W_k(\sigma)$, if $\sigma' \ne
\bot$ then $\sigma'(x)$ is eve. In this case, we know that $W_k(\sigma) =
W_{k-1}(\denote{c}\sigma) = W_{k-1}(\sigma[x := x + 2])$.
Clearly, since $\sigma(x)$ is even, so is $\sigma[x := x + 2](x)$.
Thus, we can use our induction hypothesis to conclude that if $\sigma' \ne \bot$
then $\sigma'(x)$ is even, finishing our induction step.

\problem{
Recall that we obtained the least fixed point of a monotone
and continuous function $F : D \to D$ by starting with $\bot = F^0(\bot)$ and
repeatedly applying $F$ to the result. Let's explore what happens if we do not
start with $\bot$ but with some other element $x \in D$.
\begin{enumerate}
  \item What is a necessary and sufficient condition for the sequence $F^i(x)$
  to be a chain, that is $x \sqsubseteq F(x) \sqsubseteq F^2 (x) \ldots$ ?

  \item Prove that if the condition above holds then $\bigsqcup_i F^i (x)$ is a
  fixed-point and furthermore it is the least of all fixed-points that are
  $\sqsupseteq$ than $x$.

  \item Find a necessary and sufficient condition for the $\bigsqcup_i F^i (x)$ to be
the overall least-fixed point.
\end{enumerate}
}



\end{document}