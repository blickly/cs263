\documentclass{article}
\usepackage{amsmath}
\usepackage{amsthm}
%\usepackage{fullpage}
\usepackage{stmaryrd}
\usepackage{graphicx}
\usepackage{algorithmic}
\usepackage{wrapfig}

\title{CS263. Final Exam}
\author{Ben Lickly}
\date{April 27, 2010}

\newcommand{\problem}[1]
{\subsubsection*{} %\fbox{\parbox{\textwidth}{
\vspace{-16pt} \section{} \vspace{-22pt} \qquad
#1%}}
\bigskip \bigskip
}

\newcommand{\while}[2]{\operatorname{while}\, #1\ \operatorname{do}\ #2}
\newcommand{\ifthen}[3]{\operatorname{if}\, #1\ \operatorname{then}\ #2\ \operatorname{else}\ #3}
\newcommand{\denote}[1]{\llbracket #1 \rrbracket}
\newcommand{\proves}{\vdash}
\newcommand{\axiomatic}[3]{\{#1\}\ #2\ \{#3\}}
\newcommand{\meet}{\bigwedge}
\newcommand{\powerset}{\mathcal{P}}
\newcommand{\glb}{\operatorname{glb}}

% New language commands
\newcommand{\dood}[1]{\operatorname{do}\, #1\ \operatorname{od}}
\newcommand{\callP}{\operatorname{call} P}
\newcommand{\localin}[2]{\operatorname{local}\, #1\ \operatorname{in}\ #2}

\begin{document}
\maketitle

\problem{Consider the following nondeterministic language of guarded commands.
}

Note: In the problem statement, $\dood{e}{c}$ is referred to as an ``unusual
construct'', although the semantics seem to be identical to that of a
traditional while.  I hope this was an attempt at humor on the part of the
writer and not a misinterpretation on my part.

\subsection{Static Semantics}

    Since we can distinguish the types of all expressions syntactically, we can
    actually break all typing judgements into three types: integer expressions,
    boolean expressions, and commands.  We will use the symbols
    $\proves_{int}$,$\proves_{bool}$, and $\proves_{com}$, for these,
    respectively.
\subsubsection*{Integers}
\[\frac{}
{\Gamma \proves_{int} n}
\]
\[\frac{x \in \Gamma}
{\Gamma \proves_{int} x}
\]
\[\frac{}
{\Gamma \proves_{int} g}
\]
\[\frac{\Gamma \proves_{int} e_1 \qquad \Gamma \proves_{int} e_2}
{\Gamma \proves_{int} e_1 + e_2}
\]

\subsubsection*{Booleans}
\[\frac{}
{\Gamma \proves_{bool} true}
\]
\[\frac{}
{\Gamma \proves_{bool} false}
\]
\[\frac{\Gamma \proves_{int} e}
{\Gamma \proves_{bool} e = 0}
\]
\[\frac{\Gamma \proves_{int} e}
{\Gamma \proves_{bool} e \ge 0}
\]

\subsubsection*{Commands}
\[\frac{}
{\Gamma \proves_{com} skip}
\]
\[\frac{\Gamma \proves_{com} c_1 \qquad \Gamma \proves_{com} c_2}
{\Gamma \proves_{com} c_1; c_2}
\]
\[\frac{\Gamma \proves_{int} x \qquad \Gamma \proves_{int} e}
{\Gamma \proves_{com} x := e}
\]
\[\frac{\Gamma \proves_{bool} e \qquad \Gamma \proves_{com} c}
{\Gamma \proves_{com} e \rightarrow c}
\]
\[\frac{\Gamma \proves_{com} e_1 \rightarrow c_1 \quad \dots \quad
        \Gamma \proves_{com} e_n \rightarrow c_n}
{\Gamma \proves_{com} e_1 \rightarrow c_1 \# \dots \# e_n \rightarrow c_n}
\]
\[\frac{\Gamma \proves_{com} e \rightarrow c}
{\Gamma \proves_{com} \dood{e \rightarrow c}}
\]
\[\frac{}
{\Gamma \proves_{com} \callP }
\]
\[\frac{\Gamma,x \proves_{com} c}
{\Gamma \proves_{com} \localin{x}{c}}
\]

\subsection{Dynamic Semantics}
\newcommand{\bigstep}[3]{\langle #1, #2 \rangle \Downarrow #3}
\newcommand{\smallstep}[4]{\langle #1, #2 \rangle \rightarrow \langle #3, #4 \rangle}

    For the operational semantics, we first need to formalize the notion of
    state.  We will use states that are mappings from labels to integers. We also note that the intial state may have an arbitrary mapping for $g$.  i.e. For any $n$, we may have $\sigma_0(g) = n$.
\subsubsection*{Expressions}
\[\frac{}
{\bigstep{n}{\sigma}{ n}}
\]
\[\frac{}
{\bigstep{x}{\sigma}{\sigma(x)}}
\]
\[\frac{\bigstep{e_1}{\sigma}{n_1} \qquad \bigstep{e_2}{\sigma}{n_2}}
{\bigstep{e_1 + e_2}{\sigma}{n_1 + n_2}}
\]
\[\frac{\bigstep{e_1}{\sigma}{n_1} \qquad \bigstep{e_2}{\sigma}{n_2}}
{\bigstep{e_1 * e_2}{\sigma}{n_1 * n_2}}
\]

\[\frac{}
{\bigstep{true}{\sigma}{true}}
\]
\[\frac{}
{\bigstep{false}{\sigma}{false}}
\]
\[\frac{\bigstep{e}{\sigma}{0}}
{\bigstep{e = 0}{\sigma}{ true }}
\]
\[\frac{\bigstep{e}{\sigma}{n} \qquad n \ne 0}
{\bigstep{e = 0}{\sigma}{false}}
\]
\[\frac{\bigstep{e}{\sigma}{n} \qquad n \ge 0}
{\bigstep{e \ge 0}{\sigma}{true}}
\]
\[\frac{\bigstep{e}{\sigma}{n} \qquad n < 0}
{\bigstep{e \ge 0}{\sigma}{false}}
\]


\subsubsection*{Commands}
% FIXME: Define contexts and redexes
\[\frac{}
{\smallstep{skip;c}{\sigma}{c}{\sigma}}
\]
\[\frac{\bigstep{e}{\sigma}{n}}
{\smallstep{x := e}{\sigma}{skip}{\sigma[x \gets n]}}
\]
\[\frac{\bigstep{e}{\sigma}{true}}
{\smallstep{e \rightarrow c}{\sigma}{c}{\sigma}}
\]
\[\frac{\bigstep{e_i}{\sigma}{true} \qquad i \in \{1,\dots,n\}}
{\smallstep{e_1 \rightarrow c_1 \# \dots e_n \rightarrow c_n}{\sigma}{c_i}{\sigma}}
\]
\[\frac{\bigstep{e}{\sigma}{false}}
{\smallstep{\dood{e}{c}}{\sigma}{skip}{\sigma}}
\]
\[\frac{\bigstep{e}{\sigma}{true}}
{\smallstep{\dood{e}{c}}{\sigma}{c;\dood{e}{c}}{\sigma}}
\]
\[\frac{}
{\smallstep{\callP}{\sigma}{c_P}{\sigma}}
\]
%FIXME: Where are the contexts and redexes?
\[\frac{x \in \sigma}
{\smallstep{\localin{x}{c}}{\sigma}{c; x := \sigma(x) }{\sigma[x \gets n]}}
\]
\[\frac{x \not\in \sigma}
{\smallstep{\localin{x}{c}}{\sigma}{c}{\sigma[x \gets n]}}
\]

\subsection{Weakest Precondition}
\newcommand{\weakest}[2]{\operatorname{wp}(#1, #2)}
\[
\weakest{skip}{B} = B
\]
\[
\weakest{c_1;c_2}{B} = \weakest{c_1}{\weakest{c_2}{B}}
\]
\[
\weakest{x := e}{B} = [e/x]B
\]
\[
\weakest{e \to c}{B} = e \implies \weakest{c}{B}
\]
\[
\weakest{e_1 \rightarrow c_1 \# \dots \# e_n \rightarrow c_n}{B}
= \bigwedge_{i=1}^n e_i \implies \weakest{c_i}{B}
\]
\[
\weakest{\localin{x}{c}}{B}
= \forall x: \weakest{c}{B}
\]

For the looping construct and function call, we will need to define the weakest precondition in terms of a fixed point:
% FIXME: loop and callP
\[
\weakest{\dood{e}{c}}{B} = \wedge_{k \ge 0} W_k
\]

If the call is not recursive, we can simply replace the call site with $c_P$, as in:
\[
\weakest{\callP}{B} = \weakest{c_P}{B}
\]
If, on the other hand, the call to $P$ is recursive, then this would be a recursive definition.  Instead, let us define finite approximations, where $Inline_i$ is defined as follows:
\[
Inline_i = \begin{cases}
  false \to skip &\text{if } i=0 \\
  [Inline_{i-1}\/\callP]c_P &\text{if } i > 0
\end{cases}
\]
Now let $C(\sigma)$ be defined by the limit:
\[
\begin{cases}
  false &\text{if } \exists i : \wp{Inline_i, B} = false
  true &\text{otherwise}.
\end{cases}
\]
\[
\weakest{\callP}{B} = C
\]

\problem{Consider an arbitrary imperative language of commands $c$. What are
the relationships between the following pairs of predicates?}

\problem{Consider the IMP command $c \equiv \while{b}{x := e}$.
You will have to prove for this command soundness and completeness results for
the verification condition.
}

Before, we do either of the proofs, let's prove some preliminary facts.
First, the definition of a loop invariant is something that is true
each time $b$ is executed. (Since expressions are side-effect free, it doesn't
matter whether we check before or after $b$ is evaluated).
Second, given a loop invariant, call it $I$, and some postcondition, call it $B$,
let us consider what the verification condition for our command would be:
\[
VC(c, B) = I \wedge \left( \forall x : I \implies (b \implies [e/x]I \wedge \neg b \implies B) \right)
\]

\subsection{Soundness}
For any choice of the loop invariant, the verification condition for this
command is always stronger than the weakest precondition.
\[
\forall B \forall I : VC_I(c, B) \implies WP(c, B)
\]

\subsection{Completeness}
There is a choice of the loop invariant for which the verification condition is
equivalent with the weakest precondition.
\[
\forall B \exists I : VC_I(c, B) \Leftrightarrow WP(c, B)
\]

Since we have already proved the forward direction in the previous section, it suffices to prove
\[
\forall B \exists I : VC_I(c, B) \Leftarrow WP(c, B)
\]



\end{document}
