\documentclass{article}
\usepackage{amsmath}
\usepackage{amsthm}
%\usepackage{fullpage}
\usepackage{stmaryrd}

\title{CS263: HW \#5}
\author{Ben Lickly}
\date{March 9, 2010}

\newcommand{\problem}[1]
{\subsubsection*{} %\fbox{\parbox{\textwidth}{
\vspace{-16pt} \section{} \vspace{-22pt} \qquad
#1%}}
\bigskip \bigskip
}

\newcommand{\while}[2]{\operatorname{while}\, #1\ \operatorname{do}\ #2}
\newcommand{\ifthen}[3]{\operatorname{if}\, #1\ \operatorname{then}\ #2\ \operatorname{else}\ #3}
\newcommand{\denote}[1]{\llbracket #1 \rrbracket}
\newcommand{\proves}{\vdash}
\newcommand{\axiomatic}[3]{\{#1\}\ #2\ \{#3\}}
\newcommand{\meet}{\bigwedge}
\newcommand{\powerset}{\mathcal{P}}
\newcommand{\glb}{\operatorname{glb}}

\begin{document}
\maketitle

\problem{In the abstract interpretation example with the factorial from
the lecture, you can see that the analysis was rather imprecise. How can you
improve the result of this analysis by:
\begin{enumerate}
  \item Changing the factorial program but not changing the analysis, and
  \item Changing the abstract interpretation setup but without changing the
factorial program.
\end{enumerate}
Give separate solutions for each of the above ways to address the problem.
}

\problem{In the lecture we have seen how to define the $\alpha$ and $\gamma$
functions starting from an arbitrary abstraction function $\beta$. Another way to
start is by choosing the function $\gamma$ first. Given $\gamma$, write the
definition of $\alpha$ and prove that $\alpha$ is monotonic and forms a Gallois
connection with $\gamma$. In the process of doing this you will realize that this
does not work with an arbitrary $\gamma$. What are the necessary conditions that
$\gamma$ must satisfy? Hint: My solution for $\alpha$ involves a
greatest-lower-bound operation.
}

Clearly, in order to satisfy the first requirement of a Gallios connection,
$\gamma$ must be monotonic.  It also turns out that the converse is also true. 
Taken together, these require the following restriction of
$\gamma : Abs \to \powerset(C)$
\[
\forall n, m \in Abs, n \le m \iff \gamma(n) \subseteq \gamma(m)
\]

Then we can define $\alpha : \powerset(C) \to Abs$ as follows:
\[
\alpha(S) =  \glb \left\{ a | \gamma(a) \supseteq S \right\}
\]

Claim: This definition of $\alpha$ forms a Gallios connection with $\gamma$.
\begin{proof}
Let $\gamma$ be an arbitrary function satifying our constraints, and $\alpha$
be as in our definition.

Subclaim 1: $\alpha$ and $\gamma$ are monotonic
\begin{proof}
$\gamma$ is monotonic trivially by our restriction, so we only need to prove
that $\alpha$ is monotonic.
Given $S_1 \subseteq S_2 \subseteq \powerset(C)$, clearly
$S' \supseteq S_2 \implies S' \supseteq S_1$.
Thus, 
$
\left\{ a | \gamma(a) \supseteq S_2 \right\} \supseteq \left\{ a | \gamma(a)
\supseteq S_1 \right\}.
$
 Thus,
$\gamma(S_2) \supseteq \gamma(S_1)$.
\end{proof}

Subclaim 2: $\alpha(\gamma(a)) = a$ for all $a \in Abs$
\begin{proof}
First, choose an arbitrary $a \in Abs$.
By the definitions of $\alpha$ and $\gamma$, we have that
\[
\alpha(\gamma(a)) = \glb \left\{ b | \gamma(b) \supseteq \gamma(a) \right\}
\]
By our restriction on $\gamma$, this is the same as:
\[
\glb \left\{ b | b \ge a \right\}
\]
But since $Abs$ is a lattice, this is the same as $a$.
\end{proof}

Subclaim 3: $\gamma(\alpha(S)) \supseteq S$ for all $S \in \powerset(C)$
\begin{proof}
Let $S$ be an arbitrary subset of $C$.
By our definitions of $\alpha$ and $\gamma$, we have
\[
\gamma(\alpha(S)) =  \gamma( \glb \left\{ a | \gamma(a) \supseteq S \right\} )
\]
% FIXME: I don't even know where to continue from here!!
\end{proof}

\end{proof}

\end{document}