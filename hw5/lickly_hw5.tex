\documentclass{article}
\usepackage{amsmath}
%\usepackage{fullpage}
\usepackage{stmaryrd}

\title{CS263: HW \#5}
\author{Ben Lickly}
\date{March 9, 2010}

\newcommand{\problem}[1]
{\subsubsection*{} %\fbox{\parbox{\textwidth}{
\vspace{-16pt} \section{} \vspace{-22pt} \qquad
#1%}}
\bigskip \bigskip
}

\newcommand{\while}[2]{\operatorname{while}\, #1\ \operatorname{do}\ #2}
\newcommand{\ifthen}[3]{\operatorname{if}\, #1\ \operatorname{then}\ #2\ \operatorname{else}\ #3}
\newcommand{\denote}[1]{\llbracket #1 \rrbracket}
\newcommand{\proves}{\vdash}
\newcommand{\axiomatic}[3]{\{#1\}\ #2\ \{#3\}}

\begin{document}
\maketitle

\problem{In the abstract interpretation example with the factorial from
the lecture, you can see that the analysis was rather imprecise. How can you
improve the result of this analysis by:
\begin{enumerate}
  \item Changing the factorial program but not changing the analysis, and
  \item Changing the abstract interpretation setup but without changing the
factorial program.
\end{enumerate}
Give separate solutions for each of the above ways to address the problem.
}

\problem{In the lecture we have seen how to define the $\alpha$ and $\gamma$
functions starting from an arbitrary abstraction function $\beta$. Another way to
start is by choosing the function $\gamma$ first. Given $\gamma$, write the
definition of $\alpha$ and prove that $\alpha$ is monotonic and forms a Gallois
connection with $\gamma$. In the process of doing this you will realize that this
does not work with an arbitrary $\gamma$. What are the necessary conditions that
$\gamma$ must satisfy? Hint: My solution for $\alpha$ involves a
greatest-lower-bound operation.
}

\end{document}