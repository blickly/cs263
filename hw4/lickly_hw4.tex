\documentclass{article}
\usepackage{amsmath}
%\usepackage{fullpage}
\usepackage{stmaryrd}

\title{CS263: HW \#4}
\author{Ben Lickly}
\date{February 26, 2010}

\newcommand{\problem}[1]
{\subsubsection*{} %\fbox{\parbox{\textwidth}{
\vspace{-16pt} \section{} \vspace{-22pt} \qquad
#1%}}
\bigskip \bigskip
}

\newcommand{\while}[2]{\operatorname{while}\, #1\ \operatorname{do}\ #2}
\newcommand{\ifthen}[3]{\operatorname{if}\, #1\ \operatorname{then}\ #2\ \operatorname{else}\ #3}
\newcommand{\denote}[1]{\llbracket #1 \rrbracket}
\newcommand{\proves}{\vdash}
\newcommand{\axiomatic}[3]{\{#1\}\ #2\ \{#3\}}

\begin{document}
\maketitle

\problem{
Prove using Hoare rules the following property: For any boolean
command $b$, if we start the command $\while{b}{x := x + 2}$ in a state in
which $x$ is even, and if the command terminates, it terminates in a state in
which $x$ is even.
}
First, we want to prove that incrementing $x$ by 2 will maintain its evenness.
We will start with the Hoare rule for assigment:
\[
\frac{ }
{\proves \axiomatic{A[E/x]}{x := x + 2}{A}}
\]
Here, we will use $\exists k, 2k = x$ for $A$
and $x+2$ for $E$, giving us:
\[
\proves \axiomatic{\exists k, 2k = x + 2}{x := x + 2}{\exists k, 2k = x}
\]
We can simplify the precondition, noting that if there exists a $k$ for which
$2k = x +2$, then there also exists a $k'$ (namely $k-1$) for which $2k' = 2$.
Renaming $k'$ to $k$, we get just:
\[
\proves \axiomatic{\exists k, 2k = x}{x := x + 2}{\exists k, 2k = x}
\]
From now on, we will refer to $\exists k, 2k = x$ as just $Even$.
We can use the rule of consequence to derive
\[
\proves \axiomatic{Even \wedge b}{x := x + 2}{Even}
\]
We will now use the rule for while:
\[
\frac{\proves \axiomatic{A \wedge b}{c}{A}}
{\proves \axiomatic{A}{\while{b}{c}}{A \wedge \neg b}}
\]
Here, we will use $A = Even$ again, to derive $Even \wedge \neg b$.
Finally, we can use the rule of consequence one more time to derive $Even$.

\problem{Consider the following alternate Hoare rule for $while$:
\[
\frac{\proves \axiomatic{A}{c}{b \implies A \wedge \neg b \implies B}}
{\proves \axiomatic{b \implies A \wedge \neg b \implies B}{\while{b}{c}}{B}}
\]
Show that the system of axioms remains complete if we replace the old
rule for while with this one. You must show that any derivation that uses
the old rule for while can be written with this rule instead.
}

New rule:
\[
\frac{\proves \axiomatic{A'}{c}{b \implies A' \wedge \neg b \implies B'}}
{\proves \axiomatic{b \implies A' \wedge \neg b \implies B'}{\while{b}{c}}{B'}}
\]

Now let $A' = A \wedge b$ and $B' = A \wedge \neg b$.
This gives us
\[
\frac{\proves \axiomatic {A \wedge b} {c}
  {b \implies (A \wedge b) \wedge \neg b \implies (A \wedge \neg b))}}
{\proves \axiomatic
  {b \implies (A \wedge b) \wedge \neg b \implies (A \wedge \neg b))}
  {\while{b}{c}} {A \wedge \neg b}}
\]
Let us consider the expression 
\[
b \implies (A \wedge b) \wedge \neg b \implies (A \wedge \neg b))
\]
By the definition of implication, this is equivalent to.
\[
(\neg b \vee (A \wedge b)) \wedge (b \vee (A \wedge \neg b)))
\]
We can distribute here to get
\[
(\neg b \vee A) \wedge (\neg b \vee b) \wedge (b \vee A) \wedge (\neg b \vee b)
\]
The $\neg b \vee b$s are vacuously true, so we can remove them.  That leaves us
with:
\[
(\neg b \vee A) \wedge (b \vee A)
\]
which we can cancel away to get just $A$.  Using this simplification, we can
rewrite our new rule as:
\[
\frac{\proves \axiomatic{A \wedge b}{c}{A}}
{\proves \axiomatic{A}{\while{b}{c}}{A \wedge \neg b}}
\]

But this is exactly the same as our old rule! Thus the we can use our new rule
in place of the old.


 \problem{Consider the following alternate Hoare rule for
$while$:
\[
\frac{\proves \axiomatic{A \wedge b}{c}{A}}
{\proves \axiomatic{A}{\while{b}{c}}{A}}
\]
This rule is not complete. Give a counterexample and a short justification.
}

\[\axiomatic{true}{\while{true}{skip}}{false}\] is not provable with this rule. 
This is because this fact relies on $true$ never being false, which normally
gives us a much stronger post-condidition.  We have lost this here by removing
the $\neg b$ from the final post-condidition.

\problem{Consider now another Hoare rule for $while$:
\[
\frac{\proves \axiomatic{A}{c}{A}}
{\proves \axiomatic{A}{\while{b}{c}}{A \wedge \neg b}}
\]
This rule is also incomplete. Give a counterexample and a short justification.
}

\[\axiomatic{x=0}{\while{false}{x:=x+1}}{x=0}\] is not provable with this rule.
This is because it requires too much of its proof of $c$.  It requires that 
$A$ never be changed by $c$, whereas it really only matters when $b$ is also
true.

\end{document}