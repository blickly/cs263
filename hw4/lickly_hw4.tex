\documentclass{article}
\usepackage{amsmath}
%\usepackage{fullpage}
\usepackage{stmaryrd}

\title{CS263: HW \#4}
\author{Ben Lickly}
\date{February 26, 2010}

\newcommand{\problem}[1]
{\subsubsection*{} %\fbox{\parbox{\textwidth}{
\vspace{-16pt} \section{} \vspace{-22pt} \qquad
#1%}}
\bigskip \bigskip
}

\newcommand{\while}[2]{\operatorname{while}\, #1\ \operatorname{do}\ #2}
\newcommand{\ifthen}[3]{\operatorname{if}\, #1\ \operatorname{then}\ #2\ \operatorname{else}\ #3}
\newcommand{\denote}[1]{\llbracket #1 \rrbracket}
\newcommand{\proves}{\vdash}
\newcommand{\axiomatic}[3]{\{#1\}\ #2\ \{#3\}}

\begin{document}
\maketitle

\problem{
Prove using Hoare rules the following property: For any boolean
command $b$, if we start the command $\while{b}{x := x + 2}$ in a state in
which $x$ is even, and if the command terminates, it terminates in a state in
which $x$ is even.
}

\problem{Consider the following alternate Hoare rule for $while$:
\[
\frac{\proves \axiomatic{A}{c}{b \implies A \wedge \neg b \implies B}}
{\proves \axiomatic{b \implies A \wedge \neg b \implies B}{\while{b}{c}}{B}}
\]
    Show that the system of axioms remains complete if we replace the old
rule for while with this one. You must show that any derivation that uses
the old rule for while can be written with this rule instead.
}

\problem{Consider the following alternate Hoare rule for $while$:
\[
\frac{\proves \axiomatic{A \wedge b}{c}{A}}
{\proves \axiomatic{A}{\while{b}{c}}{A}}
\]
This rule is not complete. Give a counterexample and a short justification.
}

\[\axiomatic{true}{\while{true}{skip}}{false}\] is not provable with this rule. 
This is because this fact relies on $true$ never being false, which normally
gives us a much stronger post-condidition.  We have lost this here by removing
the $\neg b$ from the final post-condidition.

\problem{Consider now another Hoare rule for $while$:
\[
\frac{\proves \axiomatic{A}{c}{A}}
{\proves \axiomatic{A}{\while{b}{c}}{A \wedge \neg b}}
\]
This rule is also incomplete. Give a counterexample and a short justification.
}

\[\axiomatic{x=0}{\while{false}{x:=x+1}}{x=0}\] is not provable with this rule.
This is because it requires too much of its proof of $c$.  It requires that 
$A$ never be changed by $c$, whereas it really only matters when $b$ is also
true.

\end{document}