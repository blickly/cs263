\documentclass{article}
\usepackage{amsmath}
\title{CS263: HW \#1}
\author{Ben Lickly}
\date{February 2, 2010}

\begin{document}
\maketitle
\begin{enumerate}
  \item Comment on some aspect from Hoare's papers ``Hints on
  programming-language design'' that relates to your programming experience. Try
  to provide additional evidence in favor of one his points and against one of
  his points. A couple of paragraphs should be enough.
  
  
  \item Let $X$ and $Y$ be sets. Show that there is a 1-1 correspondence bewteen
  the sets $X \rightarrow P(Y)$ and $P(X \times Y )$. This correspondence will
  allow us to use functional notation for certain sets.
  
  Let $S \in P(X \times Y)$. i.e. $S$ is an arbitrary set that contains elements
  of the form $(x,y)$, where $x \in X$ and $y \in Y$.  Note that this set
  defines a partial function from $f : X \rightarrow P(Y)$, by the following
  construction:
  \[
  f(x) = \{ y : (x,y) \in S \}
  \]
  Note that in the case that there is no $y$ for which $(x,y) \in S$, this
  simply means that $f(x) = \emptyset$.
  
  \item Division?
  
  \item Extend the operational semantics of the IMP language with a ``do c
  until(e)'' construct. Show the extension both for the big-step semantics and
  for the contextual small-step semantics.
  \newcommand{\dountil}[2]{\operatorname{do} #1 \operatorname{until}(#2)}
  
  Big-step semantics:
  \[\frac{
  \langle c, \sigma \rangle \Downarrow \sigma'   \qquad
  \langle e, \sigma' \rangle \Downarrow true }
  % --------------------------------------
  {\langle \dountil{c}{e}, \sigma \rangle \Downarrow \sigma'}
  \]

  \[\frac{
  \langle c, \sigma \rangle \Downarrow \sigma'   \quad
  \langle e, \sigma' \rangle \Downarrow false    \quad
  \langle \dountil{c}{e}, \sigma' \rangle \Downarrow \sigma'' }
  % --------------------------------------
  {\langle \dountil{c}{e}, \sigma \rangle \Downarrow \sigma'' }
  \]

  \item Consider the IMP language with a new command construct ``let x = e in
  c''.
  \newcommand{\letin}[2]{\operatorname{let} x = #1 \operatorname{in} #2}
  
Big-step semantics:
  \[\frac{
  \langle e, \sigma \rangle \Downarrow n    \qquad
  \langle c, \sigma[x \rightarrow n] \rangle \Downarrow \sigma' }
  % --------------------------------------
  {\langle \letin{e}{c}, \sigma \rangle \Downarrow
          \sigma'[x \rightarrow \sigma(x)]}
  \]
\end{enumerate}
\end{document}

