\documentclass{article}
\usepackage{amsmath}
\title{CS263: HW \#1}
\author{Ben Lickly}
\date{February 2, 2010}

\begin{document}
\maketitle
\begin{enumerate}
  \item Comment on some aspect from Hoare's papers ``Hints on
  programming-language design'' that relates to your programming experience. Try
  to provide additional evidence in favor of one his points and against one of
  his points. A couple of paragraphs should be enough.

In the security section, I found it strange that Hoare claimed that the ``it is
impossible to guarantee that [standard and optimizing compilers] will give the
same results." This seems completely off base today, where compiler optimizations
that do not guarantee the same results are considered erroneous.  Hoare's peice
paints a picture of compiler's whose optimization settings try various syntactic
simplifications, with no respect for program semantics.

Hoare's dismissal of independent compilation seems equally quaint.  His
complaints obviously come from a time when computer resources were much more
restricted, but the conclusions he comes to about avoiding independent
compilation seem very out of touch with today's multi-million LOC systems. It
seems that linkers of his day didn't check that the types of the functions that
they were linking together matched.  He takes this as an indictment of
independent compilation in general, which seems crazy by today's standards.

The commenting conventions in ALGOL60 seem absolutely horrible.  It's a good
thing that the PL community listened to Hoare about the importance of newline
terminated comments.  I can only imagine what a pain it would be to debug a
program that compiles without complaint even when missing a semicolon.

I don't quite understand his point (3) in the procedures and parameters section:
\begin{quote}
  The ability to define side effects of function calls negates many of the
  advantages of arithmetic expressions.
\end{quote}  
Since he already made a case for logical separation of a program into its
procedures, I would have assumed that Hoare was OK with side-effecting functions.
 Someone today could certainly use the same line of reasoning to advocate pure
functional programming, but taken with Hoare's focus on efficiency, this doesn't
make sense. It feels like he has overlooked how his statement here about side
effects conflicts with his other aims at having maximal efficiency.

  \item Let $X$ and $Y$ be sets. Show that there is a 1-1 correspondence bewteen
  the sets $X \rightarrow P(Y)$ and $P(X \times Y )$. This correspondence will
  allow us to use functional notation for certain sets.
  
  Let $S \in P(X \times Y)$. i.e. $S$ is an arbitrary set that contains elements
  of the form $(x,y)$, where $x \in X$ and $y \in Y$.  Note that this set
  defines a partial function from $f : X \rightarrow P(Y)$, by the following
  construction:
  \[
  f(x) = \{ y : (x,y) \in S \}
  \]
  Note that in the case that there is no $y$ for which $(x,y) \in S$, this
  simply means that $f(x) = \emptyset$.
  
  \item Consider the IMP language discussed in class, with the expression
  sub-language extended with a division operator. Explain what changes must be
  made to the operational semantics (big-step only).
  
  An additional rule would need to be added for division:
  \[\frac{
  \langle e_1, \sigma \rangle \Downarrow n_1   \qquad
  \langle e_2, \sigma' \rangle \Downarrow n_2  \qquad
  n_2 \ne 0 }
  % --------------------------------------
  {\langle e_1 / e_2, \sigma \rangle \Downarrow n_1 / n_2}
  \]
  
  \item Extend the operational semantics of the IMP language with a ``do c
  until(e)'' construct. Show the extension both for the big-step semantics and
  for the contextual small-step semantics.
  \newcommand{\dountil}[2]{\operatorname{do} #1 \operatorname{until}(#2)}
  \newcommand{\ifthenelse}[3]{\operatorname{if} #1 \operatorname{then} #2 \operatorname{else} #3}
  
  Big-step semantics:
  \[\frac{
  \langle c, \sigma \rangle \Downarrow \sigma'   \qquad
  \langle e, \sigma' \rangle \Downarrow true }
  % --------------------------------------
  {\langle \dountil{c}{e}, \sigma \rangle \Downarrow \sigma'}
  \]

  \[\frac{
  \langle c, \sigma \rangle \Downarrow \sigma'   \quad
  \langle e, \sigma' \rangle \Downarrow false    \quad
  \langle \dountil{c}{e}, \sigma' \rangle \Downarrow \sigma'' }
  % --------------------------------------
  {\langle \dountil{c}{e}, \sigma \rangle \Downarrow \sigma'' }
  \]

  Small-step semantics:

  As a new redex, we need:
  \[
  \langle \dountil{c}{e}, \sigma \rangle \rightarrow
          \langle \ifthenelse{e}{c}{c;\dountil{c}{e}}, \sigma \rangle
  \]
  No additional contexts are needed.


  \item Consider the IMP language with a new command construct ``let x = e in
  c''.
  \newcommand{\letin}[2]{\operatorname{let} x = #1 \operatorname{in} #2}

Big-step semantics:
  \[\frac{
  \langle e, \sigma \rangle \Downarrow n    \qquad
  \langle c, \sigma[x \rightarrow n] \rangle \Downarrow \sigma' }
  % --------------------------------------
  {\langle \letin{e}{c}, \sigma \rangle \Downarrow
          \sigma'[x \rightarrow \sigma(x)]}
  \]

Small-step semantics:

  As a new redex, we need:
  \[
  \langle \letin{n}{c}, \sigma \rangle \rightarrow
      \langle c; x:= \sigma(x), \sigma[x \rightarrow n] \rangle
  \]

  As a new context, we need:
  \[
    \letin{H}{c}
  \]
\end{enumerate}
\end{document}

