\documentclass{article}
\usepackage{amsmath}
\title{CS263: HW \#2}
\author{Ben Lickly}
\date{February 11, 2010}

\newcommand{\problem}[1]
{\subsubsection*{} %\fbox{\parbox{\textwidth}{
\vspace{-16pt} \section{} \vspace{-22pt} \qquad
#1%}}
\bigskip \bigskip
}

\begin{document}
\maketitle
\problem{
What was the bug in the proof that well-founded induction is a sound
reasoning principle (lecture 3, slide 12)?
}

  The statement ``If $y \prec x$ then $y \in D_n$'' is not always true. Since the
  $D_n$ levels are organized by the \emph{longest} descending chain, then it
  could be the case that $y \in D_i$ for some $i < n$.
  
\problem{
Prove by induction the following statement about the operational semantics:

\newcommand{\while}[2]{\operatorname{while}\, #1\ \operatorname{do}\ #2}

For any boolean command $b$ and any initial state $\sigma$ such that $\sigma(x)$
is even, if $\langle \while{b}{x := x + 2}, \sigma \rangle \Downarrow \sigma'$
then $\sigma'(x)$ is even. Make sure you state what you induct on, what is the
base case and what are the inductive cases. Show representative cases among
the latter. Do not do a proof by mathematical induction!
}

By structural induction on the strucutre of the proof:


\end{document}

